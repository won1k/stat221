\documentclass[psamsfonts]{amsart}

%-------Packages---------
\usepackage{amssymb,amsfonts}
\usepackage[all,arc]{xy}
\usepackage{enumerate}
\usepackage{mathrsfs}

%--------Theorem Environments--------
%theoremstyle{plain} --- default
\newtheorem{thm}{Theorem}[section]
\newtheorem{cor}[thm]{Corollary}
\newtheorem{prop}[thm]{Proposition}
\newtheorem{lem}[thm]{Lemma}
\newtheorem{conj}[thm]{Conjecture}
\newtheorem{quest}[thm]{Question}

\theoremstyle{definition}
\newtheorem{defn}[thm]{Definition}
\newtheorem{defns}[thm]{Definitions}
\newtheorem{con}[thm]{Construction}
\newtheorem{exmp}[thm]{Example}
\newtheorem{exmps}[thm]{Examples}
\newtheorem{notn}[thm]{Notation}
\newtheorem{notns}[thm]{Notations}
\newtheorem{addm}[thm]{Addendum}
\newtheorem{exer}[thm]{Exercise}

\theoremstyle{remark}
\newtheorem{rem}[thm]{Remark}
\newtheorem{rems}[thm]{Remarks}
\newtheorem{warn}[thm]{Warning}
\newtheorem{sch}[thm]{Scholium}

\makeatletter
\let\c@equation\c@thm
\makeatother
\numberwithin{equation}{section}

\bibliographystyle{plain}

%--------Meta Data: Fill in your info------
\title{Problem Set 2 \\ STAT 221}

\author{Won I. Lee}

%\date{July 30, 2016}


\begin{document}
	
\maketitle

\section{Posterior and Likelihood}

The joint posterior of the model is given by the following:
$$p(\mu, \sigma^2, \log \theta|Y, w) \propto p(Y|\mu, \sigma^2, \log\theta, w) p(\mu, \sigma^2, \log\theta|w)$$
where we treat $w$ as constant and fixed (thus conditioned on) throughout the calculation. We have:
$$p(Y|\mu, \sigma^2, \log\theta, w) = \prod_{j=1}^J \prod_{n=1}^N p(Y_{jn}|\mu, \sigma^2, \log\theta_j, w_j) = \prod_{j=1}^J \prod_{n=1}^N \text{Pois}(Y_{jn}|w_j\theta_j)$$
and:
$$p(\mu, \sigma^2, \log\theta|w) = p(\log\theta|\mu,\sigma^2,w)p(\mu,\sigma^2|w) \propto \frac{1}{\sigma^2}\prod_{j=1}^J \mathcal{N}(\log\theta_j|\mu, \sigma^2)$$
Putting these together, we have:
$$p(\mu, \sigma^2, \log\theta|Y, w) \propto \frac{1}{\sigma^2} \prod_{j=1}^J \mathcal{N}(\log\theta_j|\mu, \sigma^2)\prod_{n=1}^N \text{Pois}(Y_{jn}|w_j\theta_j)$$

The conditional posterior is given by:
$$p(\log\theta_j|Y, w, \mu, \sigma^2) \propto p(Y_{j\cdot}|\log\theta_j, Y_{-j,\cdot}, w, \mu, \sigma^2) p(\log\theta_j|Y_{-j,\cdot}, w, \mu, \sigma^2) $$
where $Y_{j\cdot}$ indicates $Y_{j1}, \dots, Y_{jN}$ and $Y_{-j, \cdot}$ indicates all other values of $Y$ (i.e. all other columns). We note that by the presumed i.i.d. assumptions:
$$p(Y_{j\cdot}|\log\theta_j, Y_{-j,\cdot}, w, \mu, \sigma^2) = p(Y_{j\cdot}|\log\theta_j, w) = \prod_{n=1}^N\text{Pois}(Y_{jn}|w_j\theta_j)$$
and:
$$p(\log\theta_j|Y_{-j,\cdot}, w, \mu, \sigma^2) = p(\log\theta_j|\mu,\sigma^2) = \mathcal{N}(\log\theta_j|\mu, \sigma^2)$$
so putting this together:
$$p(\log\theta_j|Y, w, \mu, \sigma^2)\propto \mathcal{N}(\log\theta_j|\mu, \sigma^2) \prod_{n=1}^N \text{Pois}(Y_{jn}|w_j\theta_j)$$

The conditional posterior is unimodal, as verified by both analytical observation and numerical visualization. That is, we substituted reasonable values for the parameters and computed the posterior numerically at grid-points of $\log\theta_j$ to verify that the posterior is unimodal. It is also log-concave, since taking derivatives twice of the log-posterior according to $\log\theta_j$ yields:
$$\frac{\partial^2}{\partial \theta_j^2} \log p(\log\theta_j|Y_{-j,\cdot}, w, \mu, \sigma^2) = -\sigma^{-2} - nw_j\theta_j < 0$$

\section{Simulate Data}

Please see \texttt{wonlee\_ps2\_functions.R} to see the desired function.

\section{Evaluate Coverage (Simple)}

We first started by designing the simulation as requested, with 50 values of $Y$ for each $\theta$ value to start. After some preliminary testing, we found that with the requested parameters, i.e. $J=1000$, $N=2$, and using 50 data points, we required an average system time of approximately 1200 seconds to perform a simulation for a fixed parameter setting $\mu, \sigma^2$ and for one draw of $\log \theta$. Thus, we estimated that we could run $\approx 3$ such simulations in 1 hour of clock time. Thus, in each job we do the following.
\begin{enumerate}
\item Draw 3 values of $\theta$ (for $J = 1000$).
\item Draw 50 values of $Y$ (for $J = 1000, N = 2$).
\item Run MCMC to obtain posterior samples.
\item Compute the lower/upper bounds for our confidence interval.
\item Compute the number of times each $\theta$ value is in the interval.
\end{enumerate}
We do this for 5 times for each set of parameter values (4), resulting in 20 jobs that yield 15 draws of $\theta$ for each parameter setting and 50 draws of $Y$ for each $\theta$.

\section{Evaluate Coverage with Exposure Weights}

\section{Evaluate Coverage with Misspecification}

\section{Interpretation and Results}

\end{document}


