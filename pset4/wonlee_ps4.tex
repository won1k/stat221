\documentclass[psamsfonts]{amsart}

%-------Packages---------
\usepackage{amssymb,amsfonts}
\usepackage[all,arc]{xy}
\usepackage{enumerate}
\usepackage{mathrsfs}

%--------Theorem Environments--------
%theoremstyle{plain} --- default
\newtheorem{thm}{Theorem}[section]
\newtheorem{cor}[thm]{Corollary}
\newtheorem{prop}[thm]{Proposition}
\newtheorem{lem}[thm]{Lemma}
\newtheorem{conj}[thm]{Conjecture}
\newtheorem{quest}[thm]{Question}

\theoremstyle{definition}
\newtheorem{defn}[thm]{Definition}
\newtheorem{defns}[thm]{Definitions}
\newtheorem{con}[thm]{Construction}
\newtheorem{exmp}[thm]{Example}
\newtheorem{exmps}[thm]{Examples}
\newtheorem{notn}[thm]{Notation}
\newtheorem{notns}[thm]{Notations}
\newtheorem{addm}[thm]{Addendum}
\newtheorem{exer}[thm]{Exercise}

\theoremstyle{remark}
\newtheorem{rem}[thm]{Remark}
\newtheorem{rems}[thm]{Remarks}
\newtheorem{warn}[thm]{Warning}
\newtheorem{sch}[thm]{Scholium}

\makeatletter
\let\c@equation\c@thm
\makeatother
\numberwithin{equation}{section}

\bibliographystyle{plain}

%--------Meta Data: Fill in your info------
\title{Problem Set 4 \\ STAT 221}

\author{Won I. Lee}

%\date{July 30, 2016}


\begin{document}
	
\maketitle

\section{Implement MCMC}

Please see the file \texttt{wonlee\_mcmc.R} for the relevant implementation, according to the specifications provided.

\section{Uniform Prior for Time Point 5}

Using data from time point 5 only, we use a uniform prior over $\Lambda$ and use 10 chains of MCMC with 12000 iterations (with 2000 iterations of burn-in) to conduct inference. {\bf Figure 1} demonstrates our convergence diagnostics for each of the dimensions of $X_2$ (which is actually sampled using MCMC, whereas $X_1$ is computed from $X_2$), using each of the 10 chains.

We initially employed a Poisson proposal scheme as outlined in Tebaldi and West (1998), i.e. using independent draws:
$$X_i^* \sim Pois(\lambda_{i,t})$$
but this appeared to yield terrible results in terms of sample size, with ESS statistics of $\approx 5$ over 10000 samples. As a result, in the spirit of incremental MCMC moves as discussed in this course, we opted for a small uniform proposal density around the current $X_i$ value, namely:
$$X_i^* \sim Unif(X_{i,t-1} - \alpha, X_{i,t-1} + \alpha)$$
for some suitable value of $\alpha$.

\section{Informative Prior for Time Point 5}

\section{Boxplots for Different Priors}

\section{Uniform Prior for All Time Points}

\section{Informative Prior for All Time Points}

\section{Comparison of Posterior Intervals}

\end{document}


